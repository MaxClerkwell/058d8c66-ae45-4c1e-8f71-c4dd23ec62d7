\documentclass{dcbl/challenge}

\setdoctitle{Compiling C}
\setdocauthor{Stephan Bökelmann}
\setdocemail{sboekelmann@ep1.rub.de}
\setdocinstitute{AG Physik der Hadronen und Kerne}
\usepackage{listings}
\usepackage{caption}
\usepackage{hyperref}

\begin{document}

Assembly is a powerful programming language, but it also comes with some drawbacks. 
Neither is it easy to learn, nor is it easy to read. 
Especially handling different types of data can be hard and time-consuming.
Writing a whole operating-system in assembly seems almost impossible. 
This is what Ritchie and Thompson believed to be true as well. 
Another problem with assembly is, that it is not hardware agnostic. 
All instructions are only valid for a specific architecture, which makes it impossible to write portable code.
When Ritchie and Thompson started developing Unix, they came across the idea of not writing the code in assembly, but rather writing a recipe for a code-generator. 
This had already been done before, resulting in other languages, such as FORTRAN and BCPL.
Due to the limitations of the former languges, they decided to write a new code-generator in assembly, which could help them writing better operating-system code, which made it easier to write a new code-generator, which made it easier to write a better OS... and so on and so forth. 
This was the process of simultaneous development of a high-level-language called \texttt{C} and the corresponding operating system called \texttt{Unix}.

\section*{Exercises}
\begin{aufgabe}
    The process of translating C into an executable is called \texttt{compiling}.
    Compiling is done in multiple steps, that can be categorized into two phases: 
    \begin{enumerate}
        \item Analysis
        \item Synthesis
    \end{enumerate} 
    There are multiple steps that need to be performed. 
    Use the Internet as well as the presentation from the annotations to draw a process-diagram for the compilation process.
    Include at least the following steps:
    \begin{itemize}
        \item assembling
        \item abstract syntax-tree
        \item linking
        \item object-file
        \item preprocessing
        \item optimizing
        \item semantic-analysis 
        \item tokenization
        \item intermediate-representation
        \item syntactic-analysis 
        \item optimization-pass
        \item code-generation
        \item parse-tree
    \end{itemize}
\end{aufgabe}

\begin{aufgabe}
    There are many differnet C-compilers out there. 
    Probably the three most common are called:
    \begin{enumerate}
        \item GCC GNU Compiler Collection
        \item MSVC Microsoft Visual Studio
        \item CLANG Clang Compiler
    \end{enumerate}
    Your question is probably: Which one should I use? 
    Well, if you are running on a linux system and you are a beginner, you should probably use GCC.
    If you are running on Windows, you should probably use MSVC.
    If you are running on MacOS, you should probably use CLANG.
    But these are just rules of thumb. 
    Whatever works for you is fine. 
    Since all of them are standard-compatible-compilers and all of them are free, you can use any of them.
    In order to install \texttt{gcc} on Linux, just use the package-manager like \texttt{sudo apt update} and \texttt{sudo apt install gcc}.
    Make sure, that your installation works by calling \texttt{gcc --version}.
\end{aufgabe}

\begin{aufgabe}
    Create a new directory in your home directory called \texttt{C-programs}.
    Open a new textfile in this directory called \texttt{return.c}.
    We will create a program, that returns the value 42 to the shell.
    Write the following code:
    \begin{lstlisting}[language={C}, caption={Return 42 in C}]
        int main() { return 42; }
    \end{lstlisting}
    Compare this program with the program from worksheet \href{https://github.com/bjoekeldude/3dc16411-8fdf-438c-b8fc-aeb1fc723a22}{3dc16411-8fdf-438c-b8fc-aeb1fc723a22}.
    Instead of writing the exact instructions we need to execute, a higher-level-language gives us a grammar that can be used to express our mental model of what the program is supposed to be doing. 
    In this example, we define a mathematical function that is called \texttt{main} and takes no arguments.
    The implementation says, that this function always returns 42, so it is basically a constant function.
    When reading C-functions, try to think of $result = f(x) = implementation$.
    So in our case $result = main( ) = 42$.
    Let's fire up our compiler and see what happens!
    Save the file and close your text-editor. 
    Run \texttt{gcc return.c -o return42}. 
    This way, \texttt{gcc} takes the C-file as an input and writes the output to the file \texttt{return42}.
    Run \texttt{./return42} and look at the return value in the shell by querying the \texttt{echo \$?} command.
\end{aufgabe}

\begin{aufgabe}
    Open a new textfile in this directory called \texttt{hello.c}.
    We will create a program, that prints \textit{hello} to the shell.
    Write the following code:
    \begin{lstlisting}[language={C}, caption={Hello World in C}]
        #include <stdio.h>
        int main() { 
            printf("hello\n"); 
            return 0; 
        }
    \end{lstlisting}
    Compare this to the assembly-version again. 
    With growing programs, it gets even more obvious, how a high-level-language can help us writing programs. 
    Especially a look into the first line will be new to us. 
    The \texttt{\#include} statement is used to copy a complete file from a different location into our source-code. 
    This way, we can use well defined functions, that other people already wrote for us.
    The \texttt{stdio.h} header-file contains all the standard-input and standard-output functions.
    Something we had to write in a tedious way to get it working in assembly.
    One thing must be kept in mind here all the time: We stopped writing the program ourselves, and started writing recipies for a code-generator!
    Let us investigate the actual steps of compilation now.\\
    \begin{enumerate}
        \item \texttt{gcc -E hello.c -o hello.i} - This first step runs the preprocessor and writes the so called pure-C file into the \texttt{hello.i}.
        \item \texttt{gcc -S hello.i -o hello.s} - This second step runs the assembler and writes an assembly-file into the \texttt{hello.s}.
        \item \texttt{gcc -c hello.s -o hello.o} - This third step compiles the assembly-file into an object-file.
        \item \texttt{gcc hello.o -o hello} - This last step links the object-file into an executable-file.
    \end{enumerate}
    Take a look at the pure-C file, as well as the assembly-file.
    Use \texttt{objdump -D -s -t -x hello.o} to get an overview of the object-file.
    And run \texttt{./hello} to see the output in the shell.
    Usually you wouldn't run each step individually but just call \texttt{gcc hello.c -o hello} and \texttt{./hello}.
\end{aufgabe}


\section*{Annotations}
\begin{enumerate}
    \item Slide-Deck about Compiling: \href{https://docs.google.com/presentation/d/1L_ALmHWfUl4o5LmlR57lYcjnFoX_JdS_3fqgFlhNT6s/edit?usp=sharing}{Presentation}
\end{enumerate}

\end{document}
